\begin{resumo}

A sequência canônica de etapas de processamento de imagens inclui aquisição, pré-processamento, segmentação, extração de características (ou representação e descrição), transformação e reconhecimento. Apesar da grande variedade de métodos disponíveis, a classificação de imagens é comumente tratada apenas considerando um subconjunto destas etapas, em especial a extração de características e o reconhecimento, sendo as demais muitas vezes negligenciadas. Enquanto isso, alguns estudos encontraram evidências de que a preparação das imagens, por meio da especificação cuidadosa da aquisição, pré-processamento e segmentação, pode impactar significativamente o resultado final da classificação. Imagens obtidas de diferentes fontes, como câmeras pessoais, microscopia, telescopia e tomografia, possuem características que podem ser exploradas para melhorar a descrição dos objetos de interesse. Assim, é proposta a melhoria da classificação de imagens utilizando métodos de pré-processamento. Entre as possibilidades de melhorias estão: o uso de realce e restauração para a extração de características latentes, isto é, características não visíveis na imagem original; aprendizagem das características latentes por meio de redes neurais de convolução e máquinas de Boltzmann restritas; e o uso de geração de imagens aleatórias, de forma a promover o balanceamento de bases de dados cujo número de exemplos de cada classe é desbalanceado. Esta dissertação possui resultados preliminares que demonstram o potencial da pesquisa e pretende contribuir com a área, investigando métodos que permitam obter melhores espaços de características.


\meutodo{
	Porque deep leaning?
	Mais descritores de forma e textura (dependentes da base)
	Explicar melhor o comportamento do SMOTE x ACC
	Bases maiores
	Algoritmos genéticos para geração de imagens artificiais?
}

\end{resumo}
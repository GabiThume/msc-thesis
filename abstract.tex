\begin{abstract}

\meutodo{melhorar muito o abstract}
The canonical steps of digital image processing can be summarized as acquisition, preprocessing, segmentation, feature extraction, transformation and recognition. Althought the large number of methods available for image classification, only a subset of those steps are explored, specially related to feature extraction and recognition. The remaining steps are generally neglected. Meanwhile, some studies found evidence that image preparation --- by means of an accurate specification of the acquisition, preprocessing and segmentation steps --- can improve the classification results. Images captured from different kinds of sources like personal cameras, microscope, telescope and tomography, have features that can be explored to extend the description of objects. Therefore, we propose to improve image classification by using preprocessing methods. Among the possibilities for improvements we can highlight: using both image enhancement and restoration to extract latent features --- non-visible characteristics in the original image; latent features learning by means of convolutional neural networks and restricted Boltzmann machines; and synthesis of random artificial images, to be used for database balancing when the number of samples from each class is originally unbalanced. This research has preliminary results which points to its potential. It aims to contribute to the research area by investigating methods to improve image classification while obtaining better feature spaces.

\end{abstract}
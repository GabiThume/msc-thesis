\documentclass[12pt,a4paper,twoside]{icmc}
%\includeonly{sdl-pr} %para compilar apenas o cap1

\usepackage[top=30mm,bottom=35mm,left=25mm,right=25mm,twoside]{geometry}
\usepackage{syntax}
\usepackage{ae}
\usepackage{amssymb}
\usepackage{setspace}
\usepackage{ifthen}
\usepackage{longtable}
\usepackage{tabularx}
\usepackage[dvips]{graphicx}
% \usepackage{subfigure}
\usepackage{natbib} % reimplementation of the \cite command
% \usepackage{cite} % should not be used with natbib
\usepackage{verbatim}
\usepackage{enumerate}
\usepackage{caption}
\usepackage{colortbl}
\usepackage{color}
\usepackage{textfit}
\usepackage{bibentry}
\usepackage{hlundef}
\usepackage{slashbox}
\usepackage{multirow}
\usepackage[nottoc,notlof,notlot]{tocbibind}
\usepackage{url}                  %% usa pacote "url"
\usepackage{fancychap}
\usepackage[hidelinks]{hyperref}
\usepackage{amsmath}
\usepackage{pdfpages}
%\usepackage{titlesec}
\let\oldurl=\url
\def\url{\protect\oldurl}
\usepackage[brazil]{babel}
\usepackage[utf8]{inputenc}
\usepackage{hyphenat}
\usepackage{subcaption}
\usepackage{xargs}                      % Use more than one optional parameter in a new commands
\usepackage[pdftex,dvipsnames]{xcolor}  % Coloured text etc.
\usepackage{indentfirst}
\usepackage[colorinlistoftodos,prependcaption, textwidth=25mm, textsize=footnotesize]{todonotes}
\newcommandx{\meutodo}[2][1=]{\todo[inline, linecolor=blue,backgroundcolor=red!25,bordercolor=red,#1]{\color{red}{\bf{#2}}}}

\definecolor{cinza}{rgb}{0.5,0.5,0.5}
\newcommand{\y}{\color{black}\rule{20pt}{7pt}}
\newcommand{\x}{\hspace*{20pt}}
\renewcommand{\r}{\color{cinza}\rule{20pt}{7pt}}
% \definecolor{G6}{rgb}{0.7,0.7,0.7}
% \newcommand{\fc}{\cellcolor{G6}}

\let\cite=\citep
\clubpenalty=10000 \widowpenalty=10000 \exhyphenpenalty=10000
\hyphenpenalty=1000

\nobibliography*
\newcommand{\apud}[2]{(\@apud#1,#2\@endapud)}
\def\@apud#1,#2\@endapud{%
   \citet{#1} \textbf{apud} \citet{#2}}%
%}

%\pdfcompresslevel=9 \pdfoutput=1  %% diminui tamanho do PDF
%\urlstyle{tt}                     %% fonte das URLs será "typewriter"
%\def\UrlNoBreaks{\do\(\do\[\do\{\do\<\do\:} %% define quebras de linha

\newtheorem{definicao}{Definição}[chapter]
\newtheorem{teorema}{Teorema}[chapter]

\hyphenation{pré-pro-ces-sa-men-to over-sam-pling}
\captionsetup{justification=centerlast}

\begin{document}

\title{Geração de imagens artificiais e extração de características latentes aplicadas à classificação de imagens}

\author{Gabriela Salvador Thumé}

\titulation{\hyphenpenalty=10000 Monografia apresentada ao Instituto
de Ciências Matemáticas e de Computação -- ICMC/USP, para o Exame de
Qualificação, como parte dos requisitos para a obtenção do título de
Mestre na Área de Ciências de Computação e Matemática Computacional.} 
\advisor{Prof. Dr. Moacir Pereira Ponti Junior}
\address{USP -- São Carlos/SP}
\date{Fevereiro/2015}

\begingroup
% \maketitle

\includepdf[pages={1}]{capa.pdf}
\mbox{}
\thispagestyle{empty}
\newpage
\includepdf[pages={2}]{capa.pdf}
\mbox{}
\thispagestyle{empty}

\thispagestyle{empty}  %<__ deixa a página sem numeração
\vfill \vfill \vfill  %<-- divide a página três vezes
%\hspace*{7cm}\box{Texto que voce quer colocar}

\hspace*{7cm}
% \box{One picture is worth more than then thousand words}
\frontmatter \pagestyle{plain}

\floatplacement{table}{!ht}

\begin{resumo}

A sequência canônica de etapas de processamento de imagens inclui aquisição, pré-processamento, segmentação, extração de características (ou representação e descrição), transformação e reconhecimento. Apesar da grande variedade de métodos disponíveis, a classificação de imagens é comumente tratada apenas considerando um subconjunto destas etapas, em especial a extração de características e o reconhecimento, sendo as demais muitas vezes negligenciadas. Enquanto isso, alguns estudos encontraram evidências de que a preparação das imagens, por meio da especificação cuidadosa da aquisição, pré-processamento e segmentação, pode impactar significativamente o resultado final da classificação. Imagens obtidas de diferentes fontes, como câmeras pessoais, microscopia, telescopia e tomografia, possuem características que podem ser exploradas para melhorar a descrição dos objetos de interesse. Assim, é proposta a melhoria da classificação de imagens utilizando métodos de pré-processamento. Entre as possibilidades de melhorias estão: o uso de realce e restauração para a extração de características latentes, isto é, características não visíveis na imagem original; aprendizagem das características latentes por meio de redes neurais de convolução e máquinas de Boltzmann restritas; e o uso de geração de imagens aleatórias, de forma a promover o balanceamento de bases de dados cujo número de exemplos de cada classe é desbalanceado. Esta dissertação possui resultados preliminares que demonstram o potencial da pesquisa e pretende contribuir com a área, investigando métodos que permitam obter melhores espaços de características.


\meutodo{
	Porque deep leaning?
	Mais descritores de forma e textura (dependentes da base)
	Explicar melhor o comportamento do SMOTE x ACC
	Bases maiores
	Algoritmos genéticos para geração de imagens artificiais?
}

\end{resumo}
\begin{abstract}

\meutodo{melhorar muito o abstract}
The canonical steps of digital image processing can be summarized as acquisition, preprocessing, segmentation, feature extraction, transformation and recognition. Althought the large number of methods available for image classification, only a subset of those steps are explored, specially related to feature extraction and recognition. The remaining steps are generally neglected. Meanwhile, some studies found evidence that image preparation --- by means of an accurate specification of the acquisition, preprocessing and segmentation steps --- can improve the classification results. Images captured from different kinds of sources like personal cameras, microscope, telescope and tomography, have features that can be explored to extend the description of objects. Therefore, we propose to improve image classification by using preprocessing methods. Among the possibilities for improvements we can highlight: using both image enhancement and restoration to extract latent features --- non-visible characteristics in the original image; latent features learning by means of convolutional neural networks and restricted Boltzmann machines; and synthesis of random artificial images, to be used for database balancing when the number of samples from each class is originally unbalanced. This research has preliminary results which points to its potential. It aims to contribute to the research area by investigating methods to improve image classification while obtaining better feature spaces.

\end{abstract}

\listoffigures

\chapter*{Lista de Siglas}

\begin{tabular}{l l}
ACC  & Auto-correlograma de cor\\
BIC  & Classificação de pixels de borda e interior\\
CCV  & Vetor de coerência de cor\\
CNN  & Rede neural de convolução\\
GCH  & Histograma de cor global\\
KNN  & K-vizinhos próximos\\
MSB  & Bits mais significativos\\
RBM  & Máquina de Boltzmann restrita\\
RGB  & Sistema de cores vermelho, verde e azul\\
SMOTE  &  \textit{Synthetic Minority Over-sampling Technique} \\
\end{tabular}

\tableofcontents 
\clearpage{\pagestyle{plain}\clearpage}

\endgroup

\mainmatter

\renewcommand{\chaptermark}[1]{%

\markboth{\chaptername
\ \thechapter.\ #1}{}}  %Chapter 2. Do it now

\renewcommand{\sectionmark}[1]{%
 \markright{\thesection.\ #1}}

\chapter{Introdução}

\meutodo{}

A tarefa de classificação de imagens consiste em predizer corretamente uma imagem como pertencente a uma classe previamente determinada. 
Um exemplo prático é a classificação da imagem de um \textit{oceano} como parte de uma classe denominada \textit{praia}.
Uma das formas de definir que certa imagem pertence à uma classe é especificar todas as regras que a caracterizam. Porém, na maioria dos casos isso é impossível. Considere imagens coloridas, com três canais de cores e de tamanho $256\times256$ pixels. Cada um desses 65536 pixels pode ser representado por $256^3$ combinações discretas de cores. Essa complexidade pode ser reduzida ao utilizar métodos de extração de características, os quais visam representar uma imagem com um número menor de valores vetoriais. Utilizando-se tal representação, pode-se desenvolver programas de computador que consigam definir e identificar a qual classe pertence uma imagem, sem a necessidade de se codificar todas as regras possíveis, por meio de algoritmos de aprendizado de máquina.

Os algoritmos de aprendizado de máquina possuem a capacidade de generalização, crucial para classificar novos exemplos não contidos na base de imagens originalmente utilizada para treinamento. Assim, esses algoritmos aprendem a determinar a classe correta para imagens de entrada. Em uma etapa posterior pode-se validar esse aprendizado, aplicando o algoritmo a novos exemplos.

As aplicações de reconhecimento de padrões em imagens possuem aspectos particulares para cada aplicação. Apesar da grande variedade de extratores de características disponíveis, nem sempre é possível resolver o problema de maneira satisfatória. Isso porque existem conjuntos de características que dificultam a diferenciação entre as classes. Um dos objetivos da área de aprendizado de máquina é encontrar quais são essas características que melhor discriminam as classes.

Em muitos casos, despende-se o maior esforço dessa tarefa ao operar no espaço de características já obtidas. Comumente são utilizadas transformações do espaço ou sistemas de classificação complexos que possam lidar com as deficiências das características extraídas. No entanto, imagens obtidas de diferentes fontes, como imagens naturais, de microscopia, telescopia e tomografia, possuem características que podem ser exploradas além dos métodos clássicos. É importante investigar métodos de processamento e preparação de imagens antes da extração dessas características. O uso desses métodos pode revelar \textit{características latentes}, não visíveis nas imagens originais. O objetivo desta dissertação é encontrar tais características latentes que possam melhor descrever certas classes do problema. Um exemplo visual será apresentado na Seção \ref{sec:latentes}.

Considerando que é comum realizar a extração de características a partir da imagem original, sem preocupação com a preparação da imagem, o enfoque desse estudo é na etapa de pré-processamento, destacada na Figura~\ref{fig:fluxo}. Esta ilustra as etapas canônicas do reconhecimento de padrões desde a aquisição da imagem até sua posterior classificação. As etapas de pré-processamento e segmentação --- apresentadas em destaque --- são normalmente pouco exploradas, quando comparadas com as etapas posteriores.

Atualmente, o estado da arte de extração e classificação de imagens corresponde ao uso de redes neurais de convolução, conhecidas por CNN \cite{Schmidhuber2014}. Essas redes são compostas por camadas de neurônios que têm por objetivo aprender quais são as melhores características que diferenciam as classes de imagens. O aprendizado, nesse caso, corresponde ao ajuste dos parâmetros para reduzir a diferença entre a saída esperada -- classe verdadeira -- e a produzida. Dessa forma, tais redes aprendem quais são as características latentes nas imagens de entrada. As redes neurais são discutidas na Seção \ref{sec:deeplearning}, onde as máquinas de Boltzmann restritas (RBM) também são apresentadas. A representação das imagens de entrada, aprendida pela etapa de treinamento da RBM, será utilizada para definir quais imagens são relevantes para o aprendizado ou não. De certa forma essas técnicas produzem versões processadas das imagens de entrada, indicando que os filtros aprendidos são os que melhor diferenciam as classes de imagens.

\begin{figure}[!ht]
 \begin{center}
   \includegraphics[width=0.3\linewidth]{figuras/flow.png}
 \end{center}
 \caption[Etapas canônicas do reconhecimento de padrões desde a aquisição da imagem até sua posterior classificação.]{Etapas canônicas do reconhecimento de padrões desde a aquisição da imagem até sua posterior classificação. As etapas de pré-processamento e segmentação --- apresentadas em destaque --- são normalmente pouco exploradas, quando comparadas com as etapas posteriores. O enfoque desse estudo é dar maior atenção à etapa de pré-processamento. \textit{Fonte:~Elaborado pelo autor.}}
 \label{fig:fluxo}
\end{figure}

\enlargethispage{-\baselineskip}

O desbalanceamento de classes também se apresenta como um obstáculo para que a classificação de imagens seja satisfatória. Esse problema é caracterizado pela diferença entre o número de exemplos disponíveis para cada classe. Muitos métodos de transformação do espaço de características e de classificação assumem que as classes da base estão balanceadas, o que nem sempre é verdade. Portanto, além de realizar pré\hyp{}processamento para a extração de características latentes, também é proposto o uso de geração de imagens artificiais com o objetivo de rebalancear a base de imagens, possivelmente melhorando o modelo criado para a classificação.

De maneira sumária, {\bf esta pesquisa busca melhorar a classificação de imagens, utilizando métodos de processamento com foco na extração de características latentes e no rebalanceamento de classes.} Os resultados preliminares obtidos, posteriormente apresentados na Seção \ref{sec:resultadospreliminares}, demonstram o potencial deste trabalho.

%--------------------------------------------------------------------------------
\section{Contextualização}

O grupo de pesquisa em Visualização, Imagens e Computação Gráfica (VICG), do Instituto de Ciências Matemáticas e de Computação (ICMC), tem atuado nas áreas de apoio para a classificação de coleções de imagens. Os trabalhos do grupo estão relacionados à visualização de informação com projeções multidimensionais e árvores~\cite{Joia2011}, assim como à extração de características e classificação de coleções de imagens~\cite{Paiva2011}. No que tange o processamento de imagens digitais, \citet{Picon2011} e \citet{Ponti2013} focaram no pré-processamento para obter melhores resultados de classificação. 

\citet{Paiva2011} mostraram que os espaços de características formados por cor e textura podem ser melhorados, porém há um limite até o qual as características podem ser transformadas, ou selecionadas, de forma a garantir a discriminação entre as classes. Tal projeto atua na investigação de métodos que permitam gerar espaços de características com maior discriminação entre as classes, facilitando a classificação. 

Em outros dois trabalhos relacionados é possível ver a diferença na performance para problemas de classificação de imagens. No primeiro, os autores atingem acurácia acima de 98\% na classificação de frutas após investigar alterações nos parâmetros de aquisição, realizar pré-processamento e obter a segmentação \cite{Rocha2010}. No segundo, os autores indicam que o método utilizado para obter a imagem em escala de cinza (comumente utilizada por algoritmos de extração), pode impactar significativamente a classificação final de diversas bases de imagens~\cite{Kanan2012}.

Recentemente, \citet{Ponti2014} demonstraram que o uso de algoritmos de pré\hyp{}processamento permite ao mesmo tempo obter vetores de características mais compactos e com maior capacidade de discriminação entre classes. Esta pesquisa pretende dar continuidade a esse trabalho, ao analisar técnicas de \textit{deep learning}. Essas técnicas realizam múltiplas operações sobre imagens de entrada de forma a aprender quais operações permitem gerar características capazes de discriminar as classes \cite{bengio2009}.


% \meutodo{Na literatura \citet{Peng2014} exploraram a geração de imagens artificiais a partir de modelos 3D CAD}

%--------------------------------------------------------------------------------
\section{Relevância e hipóteses}

Conforme anteriormente mencionado, muitos aspectos influenciam a performance da classificação de coleções de imagens. É comum encontrar bases cuja extração de características é considerada difícil, onde algoritmos canônicos de extração não conseguem extrair características que diferenciem bem as classes, prejudicando sua posterior classificação. Normalmente, tenta-se lidar com as particularidades das características extraídas através de transformações no espaço de atributos ou mesmo projetando classificadores mais elaborados. Acredita-se que, ao invés disso, é importante investigar métodos de processamento e preparação de imagens antes da extração das características. Por isso, {\bf uma das hipóteses desse trabalho é que o uso desses métodos possa revelar características latentes -- não visíveis nas imagens originais -- que podem melhorar a acurácia da classificação.}


Além disso, 
% uma ampla variedade de classificadores e de transformações assume que as classes da base estão balanceadas. 
o desbalanceamento de classes é um obstáculo para uma classificação satisfatória, e por isso também será estudado.
% Outro fator de impacto em tal acurácia é o desbalanceamento de classes nas bases de dados para pesquisa. 
Em bases médicas, por exemplo, a quantidade de imagens relacionadas com uma doença rara é menor do que as imagens de pacientes sem a doença. Nessas situações, em que as imagens representam eventos importantes porém menos frequentes, o sistema de classificação pode ter problemas para lidar com a classe minoritária. {\bf A hipótese, nesse caso, é que a geração de imagens artificiais como preparação para a extração de características pode melhorar a acurácia da classificação, quando comparada à geração de exemplos artificiais no espaço de atributos.} Ou seja, gerar novas imagens artificiais — que serão posteriormente reduzidas a atributos — pode apresentar melhores resultados para a classificação do que o \textit{bootstrap} de atributos artificiais.

%--------------------------------------------------------------------------------
\section{Objetivos}

\meutodo{Destacar as contribuições}

O objetivo desta pesquisa é explorar as etapas de processamento de imagens com o intuito de melhorar a discriminação entre classes de uma coleção de imagens. As hipóteses são:

\meutodo{hipóteses repetidas?}

\begin{itemize}
\item A geração de imagens artificiais pode contribuir com o balanceamento entre classes (em se tratando de problemas de classes desbalanceadas), melhorando a acurácia de algoritmos de classificação, quando comparada à geração de exemplos artificiais no espaço de atributos;
\item O uso de métodos de pré-processamento permite a extração de características latentes que aumentem a variância entre as classes, sem aumentar, no entanto, a variância intra-classe. Melhorando, assim, a classificação.
\end{itemize}

Com o objetivo de confirmar tais hipóteses, uma das propostas é analisar as características aprendidas por uma rede neural de convolução (CNN). Essa rede permite encontrar as características mais relevantes da base de imagens, que os extratores de características canônicos não capturam. Isso porque ela possui uma hierarquia de camadas, desde a imagem original até uma etapa de classificação, com o objetivo de aprender qual o melhor processamento para as imagens de modo a melhor discriminar as classes.

Após gerar as imagens artificiais, somente as imagens relevantes serão incluidas no treinamento. Para definir quais são as imagens que acrescentam informações na base, primeiramente será treinada uma máquina de Boltzmann restrita (RBM). A partir da matriz de características aprendida (memória associativa), é possível verificar se uma imagem acrescenta informações à base ou não. Por fim, conforme descrito na Seção \ref{sec:resultadospreliminares} de resultados preliminares, será possível analisar operações simples e canônicas de pré-processamento de imagens.

Dados tais aspectos, pode-se então diferenciar os objetivos gerais e específicos:

\begin{description}
\item[Geral] \

Investigar os métodos de pré-processamento de imagens de forma a preparar uma coleção de imagens para a extração de características. Com isso, espera-se ao mesmo tempo obter características latentes e balancear o número de instâncias de diferentes classes.

\item[Específicos] \

  \begin{itemize}
      \item Analisar o impacto da utilização de métodos canônicos de pré-processamento, como filtragem, adição de ruído e mistura, na classificação de bases de imagens. Investigar, também, o aprendizado de redes neurais de convolução, com o objetivo de observar quais são as características relevantes ao treiná-la com bases de imagens bem discriminadas em termos de cor, textura e forma;

      % obter alterações na imagem que permitam extrair as características latentes, que não são visíveis nem diretamente passíveis de extração, na imagem original;

      \item Modificar a base de imagens original de forma a tornar as características latentes visíveis. Com isso, pretende-se aumentar a variância entre as classes -- antes da extração de características e classificação -- com o auxílio dos métodos canônicos e CNN;

      \item Gerar imagens artificiais a partir das imagens pertencentes às classes minoritárias, compensando o desbalanceamento. Parte dessa tarefa já foi realizada e está descrita na Seção \ref{sec:resultadospreliminares}. O próximo passo é utilizar a matriz de características aprendida por máquinas de Boltzmann restritas para verificar se as imagens artificialmente geradas são relevantes para o aprendizado ou não, além de melhor escolher as imagens originais para a geração dessas imagens.
  \end{itemize}
\end{description}

Considerando os objetivos aqui descritos, os resultados esperados desta pesquisa estão destacados na Seção \ref{sec:resultados}.

%--------------------------------------------------------------------------------
\section{Estrutura do documento}

O conteúdo desta dissertação está estruturado como segue.

\begin{description}
\item [Capítulo~\ref{chap:revisao}:] são conceituados os principais fundamentos necessários para o desenvolvimento desta pesquisa: pré-processamento de imagens, redes neurais de convolução, máquinas de Boltzmann restritas, extração de características e desbalanceamento de classes.

\item [Capítulo~\ref{chap:proposta}:] descreve-se a metodologia de pesquisa, assim como os resultados esperados. O cronograma, com suas respectivas atividades e o tempo previsto para conclusão desta pesquisa, é destacado em seguida.

\item [Capítulo~\ref{chap:resultadospreliminares}:] apresenta e discute os resultados preliminares, ressaltando quais são os próximos passos.
% a proposta é retomada ao indicar como será resolvido o problema do desbalanceamento e a melhoria das imagens com base nas suas características latentes. 
\end{description}




\include{revisao}
\include{proposta}
% \meutodo{
%   Experimentos:
%   - Descrição das bases
%   - Descrição das ferramentas/técnicas/pacotes
%   - Descrição do protocolo
%   - Apresentação dos resultados
%   - Discussão dos resultados
% }
%
% \meutodo{
%   Variação gradual de parâmetros para identificar o que causa perdas/ganhos
% }

%%%%%%%%%%%%%%%%%%%%%%%%%%%%%%%%%%%%%%%%%%%%%%%%%%%%%%%%%%%%%%%%%%%%%%%%%%%%%%%%
\subsection{Experimento}

%%%%%%%%%%%%%%%%%%%%%%%%%%%%%%%%%%%%%%%%%%%%%%%%%%%%%%%%%%%%%%%%%%%%%%%%%%%%%%%%
\subsubsection{Base de Imagens}

 Os resultados foram obtidos utilizando a base de imagens COREL\footnote{Disponível em http://wang.ist.psu.edu/docs/related/}, composta por fotografias que representam as classes: tribos africanas, praia, construções, ônibus, dinossauros, elefantes, flores, cavalos, montanhas e tipos de comidas. São 10 classes balanceadas com 100 imagens cada. Para fins de exemplificação, foram selecionadas imagens que representam essas classes na Figura \ref{fig:corel}.

 \begin{figure}[hbpt]
 \begin{center}
   \includegraphics[width=1\linewidth]{\detokenize {figuras/exemplos_corel.png}}
 \end{center}
  \caption[Base de imagens COREL-1000.]{Base de imagens COREL-1000 utilizada. Estão representadas as 10 classes da base. \textit{Fonte: Elaborado pela autora.}}
 \label{fig:corel}
\end{figure}

%%%%%%%%%%%%%%%%%%%%%%%%%%%%%%%%%%%%%%%%%%%%%%%%%%%%%%%%%%%%%%%%%%%%%%%%%%%%%%%%
\subsubsection{Ferramentas/Técnicas}
Bokeh
%%%%%%%%%%%%%%%%%%%%%%%%%%%%%%%%%%%%%%%%%%%%%%%%%%%%%%%%%%%%%%%%%%%%%%%%%%%%%%%%
\subsubsection{Protocolo}
\begin{itemize}
\item Classes elefante e cavalo da COREL-1000;
\item 50\% de teste/treino;
\item \textbf{Geração artificial}: mistura;
\item \textbf{Quantização}: Intensidade;
\item \textbf{Extração de características}: BIC;
\item \textbf{Classificação} com KNN (K=1);
\item \textbf{Projeção multidimensional}: Redução de dimensionalidade - PCA.
\end{itemize}
%%%%%%%%%%%%%%%%%%%%%%%%%%%%%%%%%%%%%%%%%%%%%%%%%%%%%%%%%%%%%%%%%%%%%%%%%%%%%%%%
\subsubsection{Resultados e Discussão}

\begin{figure}[htbp]
  \begin{center}
    \begin{subfigure}{.32\linewidth}
      \centering
      \includegraphics[width=\linewidth]{\detokenize{figuras/cavalo-original.png}}
    \caption{Original}
    \end{subfigure}
    \begin{subfigure}{.32\linewidth}
      \centering
      \includegraphics[width=\linewidth]{\detokenize{figuras/cavalo-original2.png}}
    \caption{Original}
    \end{subfigure}
    \begin{subfigure}{.32\linewidth}
      \centering
      \includegraphics[width=\linewidth]{\detokenize{figuras/cavalo-blend.png}}
    \caption{Mistura}
  \end{subfigure}
  \end{center}
\end{figure}

Na Figura~\ref{fig:desbalanceado} está ilustrada a remoção de 50\% das imagens de treino da classe Cavalo, originalmente balanceada. Essa e as próximas projeções do espaço de características foram obtidas com a técnica para redução de dimensionalidade PCA, descrita na Seção~\ref{sec:pca}.

\begin{figure}[htbp]
  \begin{center}
    \begin{subfigure}{.49\linewidth}
      \centering
      \includegraphics[width=\linewidth]{\detokenize{figuras/original.png}}
    \end{subfigure}
    \begin{subfigure}{.49\linewidth}
      \centering
      \includegraphics[width=\linewidth]{\detokenize{figuras/desbalanceado-fixed.png}}
    \end{subfigure}
  \end{center}
  \caption{Remoção de 50\% das imagens de treino da classe Cavalo.}
  \label{fig:desbalanceado}
\end{figure}

Para esse experimento o \textit{f-score} da geração artificial de imagens teve um ganho de mais de 10\% em relação ao rebalancemento no espaço de características com o SMOTE. Para confirmar que a geração consegue inserir mais informação na classe minoritária do que apenas povoar os espaços entre os exemplos, uma execução desses métodos está demonstrada na Figura~\ref{compara_vis_treino}.

\begin{figure}[htbp]
  \begin{center}
    \begin{subfigure}{.49\linewidth}
      \centering
      \includegraphics[width=\linewidth]{\detokenize{figuras/smote-treino-fixed.png}}
    \end{subfigure}
    \begin{subfigure}{.49\linewidth}
      \centering
      \includegraphics[width=\linewidth]{\detokenize{figuras/geracao-treino-fixed.png}}
    \end{subfigure}
  \end{center}
  \caption{Comparação da geração com SMOTE versus campo visual.}
  \label{fig:compara_vis_treino}
\end{figure}

Após o treinamento realizado com as novas imagens e exemplos, o conjunto de teste foi dado ao classificação 1-NN e na Figura~\ref{fig:compara_vis_teste} o resultado está ilustrado. O interior dos quadrados representa a classe real dos exemplos, e a borda representa a classe predita pelo classificador. Nota-se que a melhoria na classificação com a geração de imagens fica visível e corresponde ao aumento do \texit{f-score}.

\begin{figure}[htbp]
  \begin{center}
    \begin{subfigure}{.49\linewidth}
      \centering
      \includegraphics[width=\linewidth]{\detokenize{figuras/smote-teste-fixed.png}}
    \end{subfigure}
    \begin{subfigure}{.49\linewidth}
      \centering
      \includegraphics[width=\linewidth]{\detokenize{figuras/geracao-teste-fixed.png}}
    \end{subfigure}
  \end{center}
  \caption{Comparação da geração com SMOTE versus campo visual.}
  \label{fig:compara_vis_teste}
\end{figure}

De uma forma geral, parece que a geração de imagens melhorou a definição da classe minoritária e foi o método que mais se assemelhou à distribuição dos dados originais. O principal problema do SMOTE pode ser verificado, porque ao realizar uma interpolação dos vetores de características originais, acaba criando exemplos em regiões do espaço que fazem parte da outra classe e também não possui capacidade de extrapolar a sua região como pode ser observado na diferença do smote no pequeno grupo de exemplos a direita do espaço versus a geração artificial em volta desses mesmo exemplos.

\begin{figure}[htbp]
  \begin{center}
    \begin{subfigure}{.49\linewidth}
      \centering
      \includegraphics[width=\linewidth]{\detokenize{figuras/smote-treino.png}}
    \end{subfigure}
    \begin{subfigure}{.49\linewidth}
      \centering
      \includegraphics[width=\linewidth]{\detokenize{figuras/geracao-treino.png}}
    \end{subfigure}
    \begin{subfigure}{.49\linewidth}
      \centering
      \includegraphics[width=\linewidth]{\detokenize{figuras/desbalanceado-treino.png}}
    \end{subfigure}
  \end{center}
  \caption{}
  \label{}
\end{figure}

É válido também visualizar a região de decisão para ver suas modificações frente aos métodos. E o que pode ser observado é que ambas técnicas a região da classe minoritária ficou melhor representada. Além disso, é possivel verificar que o SMOTE ocasionou uma certa invasão do espaço de características da classe majoritária.

\begin{figure}[htbp]
  \begin{center}
    \begin{subfigure}{.49\linewidth}
      \centering
      \includegraphics[width=\linewidth]{\detokenize{figuras/smote-teste-region.png}}
    \end{subfigure}
    \begin{subfigure}{.49\linewidth}
      \centering
      \includegraphics[width=\linewidth]{\detokenize{figuras/geracao-teste-region.png}}
    \end{subfigure}
    \begin{subfigure}{.49\linewidth}
      \centering
      \includegraphics[width=\linewidth]{\detokenize{figuras/desbalanceado-teste-region.png}}
    \end{subfigure}
  \end{center}
  \caption{Região de decisão com K-NN (K = 1)}
  \label{fig:region}
\end{figure}

Na Figura~\ref{fig:vis_images} as imagens foram utilizadas como marcadores e ficou nítido o impacto da etapa de extração de características na separação das classes.

\begin{figure}[htbp]
  \begin{center}
      \includegraphics[width=\linewidth]{\detokenize{figuras/vis-images.png}}
  \end{center}
  \caption{Visualização do impacto do descritor de características.}
  \label{fig:vis_images}
\end{figure}

A projeção forneceu uma análise intuitiva sobre o comportamento das técnicas que não é facilmente analisando quando olhado apenas os valores de acurácia ou \textit{f-score}.
%--------------------------------------------------------------------------------
% \section{Considerações iniciais}
%
% Este capítulo apresenta os resultados preliminares obtidos. Primeiramente, é apresentada a descrição do experimento realizado, ressaltando os métodos de extração de características, quantização e classificação utilizados. Além disso, o fluxo de operações para a realização do experimento é descrito. Em seguida, os resultados são propriamente ilustrados, ao indicar que a geração de imagens artificiais é promissora para o cenário de bases desbalanceadas. Por fim, as atividades futuras são destacadas.
%
% %--------------------------------------------------------------------------------
% \section{Descrição do experimento}
%
% Algumas pesquisas sobre os efeitos da sobreamostragem e geração de exemplos artificiais em dados de aprendizado de máquina já foram realizadas~\cite{Kuncheva2004,Chawla2002}. O método mais divulgado na literatura é conhecido como SMOTE (\textit{Synthetic Minority Over-sampling Technique}). Este método propõe a geração de exemplos artificiais a partir dos vetores de características originais das classes minoritárias. Não há registro conhecido de um estudo dessas técnicas em dados de informação visual para o rebalanceamento de classes.
%
% % \enlargethispage{-\baselineskip}
%
% Assim, foi proposta a geração de novas imagens a partir de operações como adição de ruído, borramento, mistura e combinação das imagens originais. Tais operações estão exemplificadas na Figura~\ref{fig:ArtificialImages}, utilizando a classe ``praia'' da base de imagens naturais COREL-1000. A partir das imagens originais --- primeira linha da figura --- são geradas imagens artificiais por meio das operações citadas, resultando nas imagens da segunda linha da figura.
%
% \vspace{25pt}
%
% \renewcommand{\tabcolsep}{0.04cm}
% \begin{figure}[!h]
%  \begin{center}
%  \begin{tabular}{ccccc}
%    \includegraphics[width=0.245\linewidth]{\detokenize {figuras/original-1.jpg}}&
%    \includegraphics[width=0.245\linewidth]{\detokenize {figuras/original-2.jpg}}&
%    \includegraphics[width=0.245\linewidth]{\detokenize {figuras/original-3.jpg}}&
%    % \includegraphics[width=0.19\linewidth]{\detokenize {figuras/original-4.jpg}}&
%    \includegraphics[width=0.245\linewidth]{\detokenize {figuras/original-5.jpg}}\\
%    \includegraphics[width=0.245\linewidth]{\detokenize {figuras/gerada-1_blur.jpg}}&
%    \includegraphics[width=0.245\linewidth]{\detokenize {figuras/gerada-2_ruido.jpg}}&
%    \includegraphics[width=0.245\linewidth]{\detokenize {figuras/gerada-3_blend.jpg}}&
%    % \includegraphics[width=0.19\linewidth]{\detokenize {figuras/gerada-4_unsharpMask.jpg}}&
%    \includegraphics[width=0.245\linewidth]{\detokenize {figuras/gerada-5.jpg}} \\
%  \end{tabular}
%  \end{center}
%   \caption[Geração de imagens artificiais para o rebalanceamento de classes.]{Geração de imagens artificiais para o rebalanceamento de classes. A partir das imagens originais mostradas na primeira linha, são geradas imagens artificiais por meio de: borramento, adição de ruído, mistura e combinação. Os resultados dessas operações estão demonstrados na segunda linha, em ordem. \textit{Fonte:~Elaborado pela autora.}}
%  \label{fig:ArtificialImages}
% \end{figure}
% \renewcommand{\tabcolsep}{0.5cm}
% \vspace{25pt}
%
%
% Os descritores de características utilizados para os resultados foram apresentados na Seção \ref{sec:extracao}. \todo{qual experimento anterior? explicar!} Considerando que em um experimento anterior o melhor resultado foi atribuído à quantização com o método de Intensidade para o extrator Haralick e MSB para os outros, apenas esses testes foram aprofundados (tópico anteriormente discutido na Seção \ref{sec:quantizacao}). Neste experimento, o classificador KNN foi utilizado, com $K=1$. Inicialmente o classificador Naive Bayes foi explorado, apresentando melhora na acurácia ao apenas replicar as imagens. Esse comportamento não é desejado em um classificador para a avaliação de rebalanceamento de classes. O código desenvolvido para esses resultados preliminares está disponível em \url{https://bitbucket.org/moacirponti/imagefeatureextraction/overview}.
%
%
% %--------------------------------------------------------------------------------
% \subsection{Fluxo de operações}
%
% Para a realização desse experimento, iniciou-se com uma base originalmente balanceada e foram realizadas as seguintes operações:
%
% \begin{enumerate}
% \item Diminuir logaritmicamente o número de imagens de uma das classes, de modo a obter uma base desbalanceada;
% \item Para cada estágio de desbalanceamento, realizar três experimentos:
% \begin{enumerate}
% \item A classificação direta, sem nenhuma operação de rebalanceamento;
% \item A operação de SMOTE, após a extração de características e antes da classificação;
% \item Rebalanceamento da classe minoritária com a geração de imagens antes da extração de características.
% \label{item}
% \end{enumerate}
% \item Extrair as características com os descritores: ACC, BIC, CCV, GCH e Haralick6; e os quantizadores: Intensidade, Gleam, Luminância e MSB;
% \item Classificar com KNN utilizando validação cruzada por \textit{repeated random-subsampling};
% \item Executar os passos de 2 a 4 no mínimo 10 vezes para cada par de extrator e quantizador;
% \item Calcular a matriz de confusão, a acurácia balanceada, a medida-F e o teste de Friedman para os resultados encontrados;
% \item Gerar os gráficos para visualização dos resultados.
% \end{enumerate}
%
% %--------------------------------------------------------------------------------
% \subsection{Geração das imagens artificiais}
%
% As etapas para a geração das imagens artificiais, passo \ref{item} da seção anterior, foram:
%
% \begin{enumerate}
% \item Particionar a classe minoritária em conjuntos de treino e teste;
% \item Selecionar uma imagem aleatoriamente do conjunto de treino;
% \item Selecionar uma operação aleatória entre: borramento, adição de ruído, \textit{unsharp mask}, mistura ou composição;
% \begin{enumerate}
% \item Caso seja selecionada a composição: encontrar uma outra imagem aleatória, selecionar um quadrante dessa imagem e novamente uma operação entre: borramento, adição de ruído, \textit{unsharp mask} ou mistura;
% \end{enumerate}
% \item Aplicar essa operação na imagem previamente selecionada e adicionar essa imagem gerada ao conjunto de treino;
% \item Repetir os passos 2 a 4 até que as classes estejam igualmente balanceadas.
% \end{enumerate}
%
% %--------------------------------------------------------------------------------
% \section{Resultados}
% \label{sec:resultadospreliminares}
%
% Este estudo preliminar apresentou evidências experimentais de que, em problemas de duas classes (apresentadas na Figura~\ref{fig:praiamontanha}), pode haver ganho estatístico da medida-F ao gerar imagens, quando comparado à geração de exemplos artificiais no espaço de atributos (ou seja, depois que as características já foram extraídas das imagens). Essa melhoria pode ser notada na Figura~\ref{fig:resultmelhor}, que apresenta a relação da medida-F com a taxa de balanceamento, utilizando: as imagens originais, a geração de exemplos com SMOTE e as imagens geradas. Para essa configuração, foi utilizado o descritor de características ACC com a conversão em escala de cinza por MSB e a operação de pré-processamento por combinação. As classes ``praia'' e ``montanha'' foram escolhidas por serem as classes que possuem maior dificultade de diferenciação, havendo alta taxa de sobreposição de intensidades de cores e texturas, conforme testes realizados.
%
% \vspace{20pt}
%
% \begin{figure}[!htb]
%  \begin{center}
%    \includegraphics[width=\linewidth]{\detokenize {figuras/praia-montanha.png}}
%  \end{center}
%   \caption[Classes ``praia'' e ``montanha'' da base de imagens COREL-1000.]{Classes ``praia'' (primeira linha) e ``montanha'' (segunda linha) da base de imagens COREL-1000. \textit{Fonte:~Elaborado pela autora.}}
%  \label{fig:praiamontanha}
% \end{figure}
%
% \vspace{10pt}
%
% \begin{figure}[!hbpt]
%  \begin{center}
% \begin{subfigure}{\textwidth}
%   \centering
%   \includegraphics[width=\linewidth]{\detokenize {figuras/resultado-melhor4.png}}
%   \caption{Original}
% \end{subfigure}
% \begin{subfigure}{\textwidth}
%   \centering
%   \includegraphics[width=\linewidth]{\detokenize {figuras/resultado-pior1.png}}
%   \caption{\textit{Unsharp masking}}
%   \label{fig:unsharp}
% \end{subfigure}
%  \end{center}
% \end{figure}
%
% \begin{figure}[htb]
%  \begin{center}
%    \includegraphics[width=\linewidth]{\detokenize {figuras/resultado-melhor4.png}}
%  \end{center}
%  \caption[Resultado obtido com a operação de combinação apresentada na Figura~\ref{fig:ArtificialImages}.]{Resultado obtido com a operação de combinação apresentada na Figura~\ref{fig:ArtificialImages}. Apresenta-se a relação da medida-F com a taxa de balanceamento utilizando: as imagens originais, a geração de exemplos com SMOTE e as imagens geradas artificialmente. \textit{Fonte:~Elaborado pela autora.}}
%  \label{fig:resultmelhor}
% \end{figure}
%
% \enlargethispage{-1cm}
%
% Também foi possível notar que algumas operações não provocaram a melhora da classificação. A operação de adição de ruído para geração artificial, a posterior extração utilizando CCV e a quantização por MSB, destacou-se como o pior resultado, apresentado na Figura~\ref{fig:resultpior}. Outros casos que não obtiveram o resultado esperado envolveram as operações de borramento e de \textit{unsharp masking}.
%
% \begin{figure}[htb]
%  \begin{center}
%    \includegraphics[width=\linewidth]{\detokenize {figuras/resultado-pior1.png}}
%  \end{center}
%  \caption[Piores resultados, obtidos com a adição de ruído.]{Piores resultados, obtidos com a adição de ruído. Apresenta-se a relação da medida-F com a taxa de balanceamento utilizando as imagens originais, o SMOTE e as imagens artificiais geradas. \textit{Fonte:~Elaborado pela autora.}}
%  \label{fig:resultpior}
% \end{figure}
%
% Após a realização dos testes, as operações que melhor se destacaram foram: utilizar todas as operações, apenas mistura e apenas composição. E as operações que resultaram em uma classificação pior do que o uso do SMOTE foram: utilizar apenas borramento, ruído ou \textit{unsharp masking}. Com o teste estatístico de Friedman foi possível verificar que o ACC foi o extrator que melhor se beneficiou das características geradas; e CCV e GCH os menos beneficiados. \enlargethispage{-\baselineskip} A Tabela \ref{tab:result} apresenta os \textit{rankings} encontrados por este teste para todas as execuções das melhores operações. O p-valor computado corresponde a $4.24E^{-11}$, assim a hipótese nula de que não há diferença entre as execuções foi rejeitada. Vale destacar que para algumas execuções, o teste de Friedman retornou o \textit{ranking}: geração artificial (1), SMOTE (2) e imagens originais (3), ou seja, sem que SMOTE e a geração artificial concorressem pela mesma posição, diferente da tabela apresentada.
%
% \begin{table}[htb]
% \centering
% \caption{Posição média dos algoritmos utilizando Friedman}
%   \begin{tabular}{c|c}
%     Algoritmos  &   Posição \\ \hline
%     Original    &   3.0000  \\
%     Smote       &   1.6136  \\
%     Artificial  &   1.3863  \\
%   \end{tabular}
%  \label{tab:result}
% \end{table}
%
% Em outro experimento, utilizou-se as cópias das imagens de treino para rebalancear, sem realizar nenhuma operação de pré-processamento (método conhecido como SRS - \textit{Simple Random Sampling}). A Figura~\ref{fig:resultcopia} mostra as respectivas medidas-F encontradas. É possível notar que a cópia dessas imagens não adiciona nenhuma informação nova para o aprendizado.
%
% \begin{figure}[htb]
%  \begin{center}
%    \includegraphics[width=\linewidth]{\detokenize {figuras/resultado-copia.png}}
%  \end{center}
%   \caption[Simples replicação de exemplos sem realizar nenhuma operação.]{Simples replicação de exemplos sem realizar nenhuma operação de pré-processamento. É possível verificar que não foi adicionada nenhuma informação relevante para o aprendizado. \textit{Fonte:~Elaborado pela autora.}}
%  \label{fig:resultcopia}
% \end{figure}
% %--------------------------------------------------------------------------------
% \section{Trabalhos futuros}
%
% O treinamento de uma rede neural de convolução~\footnote{\url{http://caffe.berkeleyvision.org/}} foi realizado, utilizando as classes ``praia'' e ``montanha'', balanceadas, da base COREL-1000. A classificação sobre este treinamento obteve $\approx 80\%$ de acurácia, enquanto que utilizando os extratores padrões foi possível atingir apenas $\approx 69\%$. Isso reforça a proposta de analisar quais são as características latentes que esse tipo de rede neural consegue extrair. Para essa análise vão ser utilizadas bases discriminadas quanto às propriedades de textura, cor e forma.
%
% Além de analisar o processamento realizado por uma rede de convolução para a classificação das imagens, uma rede RBM também será treinada para análise da sua memória associativa. As imagens artificialmente geradas foram adicionadas no conjunto de treino sem verificação da sua relevância, o que pode ter prejudicado a classificação. A memória associativa aprendida com o treinamento de uma máquina de Bolztmann restrita pode vir a auxiliar no entendimento dessas imagens como relevantes ou não. Além disso, pode servir como escolha para qual imagem original utilizar, ao invés do método aleatório utilizado nos resultados preliminares.
%
%
% %--------------------------------------------------------------------------------
% \section{Considerações finais}
%
% Com os experimentos realizados foi possível notar que a geração de imagens artificiais pode gerar novas informações para a classificação das imagens. O que indica que um estudo mais aprofundado de quais operações podem ser aplicadas nas imagens originais auxilie o cenário de bases desbalanceadas.
%
% Dessa forma, esse capítulo também apresentou as próximas tarefas a serem realizadas. Foi destacada a análise das redes de convolução para identificar quais características latentes são automaticamente extraídas. Apesar de algumas operações de pré-processamento terem gerado imagens que melhoraram a classificação, algumas não causaram melhora. Isso indica que a análise da relevância da informação contida em imagens deve melhorar esse resultado. A memória associativa, aprendida com uma máquina de Boltzmann restrita, deve ser capaz de indicar se uma determinada imagem é relevante para o aprendizado.

% 
% \meutodo{
% Responder aqui as perguntas levantadas na introdução, retomando todo o trabalho:
%
% - qual  o problema?
%
% - como se pretende resolver o desbalanceamento? o que as características latentes têm a ver com isso? e o pré-processamento?
%
% - limitações
% - quais descritores serão usados? porquê?
%
% - porquê e como será usado CNN? qual o objetivo?
%
% - quais os resultados preliminares? o que eles mostram? indicam que estamos no caminho certo?
%
% - o que se pretende fazer daqui pra frente? o que é esperado como resultado?
% }

Ficou constatado que um vetor original de $D$ dimensões pode ser reduzido a $d \approx D/4$ modifcando apenas o parâmetro de quantização e produzindo bons resultados. Outra possibilidade é utilizar esse métodos como um primeiro passo de redução e então utilizar o LPP para obter apenas 100 características que melhor representam os dados, atingindo acurácias maiores ou similares.

É importante ressaltar que realizar a quantização de imagens não requere treinamento e já faz parte de uma tarefa do pipeline de reconhecimento. Por esta razão, seu uso não aumenta o custo computacional do sistema, e ainda simplifica os passos subsequentes. Isso reduz a dimensão do vetor de características para os vetores de cor e o tempo de computação para os descritores de textura. Outra observação importante é que a quantização é usada especialmente para dados visuais, então não é um método geral de redução de dimensionalidade.

Com os experimentos realizados foi possível notar que a geração de imagens artificiais pode gerar novas informações para a classificação das imagens. Assim a geração de elementos no espaço visual provou contribuir com o balanceamento entre classes (em se tratando de problemas de classes desbalanceadas), melhorando a acurácia de algoritmos de classificação, quando comparada à geração de exemplos artificiais no espaço de atributos (i.e.\ SMOTE). Para validar a ideia da geração artificial de imagens, as características das novas imagens -- extraídas com o método BIC  -- e os exemplos resultantes da interpolação de vetores originais foram projetados no plano das imagens originais antes do desbalanceamento. Dessa forma foi possível visualizar que a geração de imagens artificiais proposta foi capaz de ocupar uma região do espaço mais abrangente do que o SMOTE. Este último, comprovadamente, possui o ponto negativo de não extrapolar os limites da classe minoritária. Ainda, está suscetível à criação de novos exemplos em regiões da classe majoritária, o que também prejudica a classificação.

% A visualização do espaço de características utilizando a técnica de análise de componentes principais (PCA) se mostrou crucial para confirmar visualmente que a melhora da acurácia da classificação se deve à melhor definição da classe minoritária.


% A projeção forneceu uma análise intuitiva sobre o comportamento das técnicas que não é facilmente analisando quando olhado apenas os valores de acurácia ou \textit{f-score}.



\section{Publicações}

\section{Trabalhos Futuros}

Ao usar imagens com reduzido número de cores (quantizadas), os métodos de extração de características baseados em orientação (HoG, SIFT), \textit{bag of visual words} e \textit{Fisher vectors}, seriam provavelmente mais esparços.

Em deep leraning, pode ser investigado se o uso de imagens quantizadas ajudaria o aprendizado de características.

Como extensão dos experimentos reportados pode ser feita a análise dos espaços encontrados para os diferentes métodos de geração artificial de imagens. Além disso, o impacto de tais métodos em diferentes extratores de características pode sugerir quais são as características latentes percebidas com cada extrator.

%--------------------------------------------------------------------------------
% \section{Trabalhos futuros}
%
% O treinamento de uma rede neural de convolução~\footnote{\url{http://caffe.berkeleyvision.org/}} foi realizado, utilizando as classes ``praia'' e ``montanha'', balanceadas, da base COREL-1000. A classificação sobre este treinamento obteve $\approx 80\%$ de acurácia, enquanto que utilizando os extratores padrões foi possível atingir apenas $\approx 69\%$. Isso reforça a proposta de analisar quais são as características latentes que esse tipo de rede neural consegue extrair. Para essa análise vão ser utilizadas bases discriminadas quanto às propriedades de textura, cor e forma.
%
% Além de analisar o processamento realizado por uma rede de convolução para a classificação das imagens, uma rede RBM também será treinada para análise da sua memória associativa. As imagens artificialmente geradas foram adicionadas no conjunto de treino sem verificação da sua relevância, o que pode ter prejudicado a classificação. A memória associativa aprendida com o treinamento de uma máquina de Bolztmann restrita pode vir a auxiliar no entendimento dessas imagens como relevantes ou não. Além disso, pode servir como escolha para qual imagem original utilizar, ao invés do método aleatório utilizado nos resultados preliminares.
%


\meutodo{
  Porque deep leaning?
  Mais descritores de forma e textura (dependentes da base)
  Explicar melhor o comportamento do SMOTE x ACC
  Bases maiores
  Algoritmos genéticos para geração de imagens artificiais?
}

\meutodo{
Atualmente, o estado da arte de extração e classificação de imagens corresponde ao uso de redes neurais de convolução, conhecidas por CNN \cite{Schmidhuber2014}. Essas redes são compostas por camadas de neurônios que têm por objetivo aprender quais são as melhores características que diferenciam as classes de imagens. O aprendizado, nesse caso, corresponde ao ajuste dos parâmetros para reduzir a diferença entre a saída esperada -- classe verdadeira -- e a produzida. Dessa forma, tais redes aprendem quais são as características latentes nas imagens de entrada. As redes neurais são discutidas na Seção \ref{sec:deeplearning}, onde as máquinas de Boltzmann restritas (RBM) também são apresentadas. A representação das imagens de entrada, aprendida pela etapa de treinamento da RBM, será utilizada para definir quais imagens são relevantes para o aprendizado ou não. De certa forma essas técnicas produzem versões processadas das imagens de entrada, indicando que os filtros aprendidos são os que melhor diferenciam as classes de imagens.
}
\meutodo{
Investigar, também, o aprendizado de redes neurais de convolução, com o objetivo de observar quais são as características relevantes ao treiná-la com bases de imagens bem discriminadas em termos de cor, textura e forma;
}
\meutodo{
Para definir quais são as imagens que acrescentam informações na base, primeiramente será treinada uma máquina de Boltzmann restrita (RBM). A partir da matriz de características aprendida (memória associativa), é possível verificar se uma imagem acrescenta informações à base ou não.
}
\meutodo{
O próximo passo é utilizar a matriz de características aprendida por máquinas de Boltzmann restritas para verificar se as imagens artificialmente geradas são relevantes para o aprendizado ou não, além de melhor escolher as imagens originais para a geração dessas imagens.
}
\meutodo{
A extração de características foi abordada, apresentando os principais descritores utilizados nesta pesquisa. A lacuna destacada é que existem características não passíveis de extração por descritores convencionais. Para isso, as redes de convolução são apresentadas, pois possuem capacidade de aprender as características relevantes das imagens de entrada. Justificando, assim, seu uso para análise das propriedades das bases de imagens. Podem também indicar possíveis operações para auxiliar na geração de imagens artificiais. }
\meutodo{
Ainda, a memória associativa aprendida por máquinas de Boltzmann restritas pode ser indicadora de quais imagens geradas adicionam informações ao aprendizado.
}
\meutodo{
Esta pesquisa pretende dar continuidade a esse trabalho, ao analisar técnicas de \textit{deep learning}. Essas técnicas realizam múltiplas operações sobre imagens de entrada de forma a aprender quais operações permitem gerar características capazes de discriminar as classes \cite{bengio2009}.
}
\meutodo{
Com o objetivo de confirmar tais hipóteses, uma das propostas é analisar as características aprendidas por uma rede neural de convolução (CNN). Essa rede permite encontrar as características mais relevantes da base de imagens, que os extratores de características canônicos não capturam. Isso porque ela possui uma hierarquia de camadas, desde a imagem original até uma etapa de classificação, com o objetivo de aprender qual o melhor processamento para as imagens de modo a melhor discriminar as classes.
}


%\appendix

\bibliographystyle{icmc2}
\bibliography{referencias}

%\appendix

%\include{apendices}

\end{document}
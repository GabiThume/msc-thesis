
% \meutodo{
% Responder aqui as perguntas levantadas na introdução, retomando todo o trabalho:
%
% - qual  o problema?
%
% - como se pretende resolver o desbalanceamento? o que as características latentes têm a ver com isso? e o pré-processamento?
%
% - quais descritores serão usados? porquê?
%
% - porquê e como será usado CNN? qual o objetivo?
%
% - quais os resultados preliminares? o que eles mostram? indicam que estamos no caminho certo?
%
% - o que se pretende fazer daqui pra frente? o que é esperado como resultado?
% }

Com os experimentos realizados foi possível notar que a geração de imagens artificiais pode gerar novas informações para a classificação das imagens. Assim a geração de elementos no espaço visual provou contribuir com o balanceamento entre classes (em se tratando de problemas de classes desbalanceadas), melhorando a acurácia de algoritmos de classificação, quando comparada à geração de exemplos artificiais no espaço de atributos (i.e.\ SMOTE).

A visualização do espaço de características utilizando a técnica de análise de componentes principais (PCA) se mostrou crucial para confirmar visualmente que a melhora da acurácia da classificação se deve à melhor definição da classe minoritária.

Para validar a ideia da geração artificial de imagens, as características das novas imagens -- extraídas com o método BIC  -- e os exemplos resultantes da interpolação de vetores originais foram projetados no plano das imagens originais antes do desbalanceamento. Dessa forma foi possível visualizar que a geração de imagens artificiais proposta foi capaz de ocupar uma região do espaço mais abrangente do que o SMOTE. Este último, comprovadamente, possui o ponto negativo de não extrapolar os limites da classe minoritária. Ainda, está suscetível à criação de novos exemplos em regiões da classe majoritária, o que também prejudica a classificação.

A projeção forneceu uma análise intuitiva sobre o comportamento das técnicas que não é facilmente analisando quando olhado apenas os valores de acurácia ou \textit{f-score}.

Como extensão dos experimentos reportados pode ser feita a análise dos espaços encontrados para os diferentes métodos de geração artificial de imagens. Além disso, o impacto de tais métodos em diferentes extratores de características pode sugerir quais são as características latentes percebidas com cada extrator.


\section{Publicações}

\section{Trabalhos Futuros}

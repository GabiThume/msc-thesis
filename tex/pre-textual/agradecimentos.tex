% ---
% Agradecimentos
% ---

\meutodo{
  Porque deep leaning?
  Mais descritores de forma e textura (dependentes da base)
  Explicar melhor o comportamento do SMOTE x ACC
  Bases maiores
  Algoritmos genéticos para geração de imagens artificiais?
}

\meutodo{
Atualmente, o estado da arte de extração e classificação de imagens corresponde ao uso de redes neurais de convolução, conhecidas por CNN \cite{Schmidhuber2014}. Essas redes são compostas por camadas de neurônios que têm por objetivo aprender quais são as melhores características que diferenciam as classes de imagens. O aprendizado, nesse caso, corresponde ao ajuste dos parâmetros para reduzir a diferença entre a saída esperada -- classe verdadeira -- e a produzida. Dessa forma, tais redes aprendem quais são as características latentes nas imagens de entrada. As redes neurais são discutidas na Seção \ref{sec:deeplearning}, onde as máquinas de Boltzmann restritas (RBM) também são apresentadas. A representação das imagens de entrada, aprendida pela etapa de treinamento da RBM, será utilizada para definir quais imagens são relevantes para o aprendizado ou não. De certa forma essas técnicas produzem versões processadas das imagens de entrada, indicando que os filtros aprendidos são os que melhor diferenciam as classes de imagens.
}
\meutodo{
Investigar, também, o aprendizado de redes neurais de convolução, com o objetivo de observar quais são as características relevantes ao treiná-la com bases de imagens bem discriminadas em termos de cor, textura e forma;
}
\meutodo{
Para definir quais são as imagens que acrescentam informações na base, primeiramente será treinada uma máquina de Boltzmann restrita (RBM). A partir da matriz de características aprendida (memória associativa), é possível verificar se uma imagem acrescenta informações à base ou não.
}
\meutodo{
O próximo passo é utilizar a matriz de características aprendida por máquinas de Boltzmann restritas para verificar se as imagens artificialmente geradas são relevantes para o aprendizado ou não, além de melhor escolher as imagens originais para a geração dessas imagens.
}
\meutodo{
A extração de características foi abordada, apresentando os principais descritores utilizados nesta pesquisa. A lacuna destacada é que existem características não passíveis de extração por descritores convencionais. Para isso, as redes de convolução são apresentadas, pois possuem capacidade de aprender as características relevantes das imagens de entrada. Justificando, assim, seu uso para análise das propriedades das bases de imagens. Podem também indicar possíveis operações para auxiliar na geração de imagens artificiais. }
\meutodo{
Ainda, a memória associativa aprendida por máquinas de Boltzmann restritas pode ser indicadora de quais imagens geradas adicionam informações ao aprendizado.
}
\meutodo{
Esta pesquisa pretende dar continuidade a esse trabalho, ao analisar técnicas de \textit{deep learning}. Essas técnicas realizam múltiplas operações sobre imagens de entrada de forma a aprender quais operações permitem gerar características capazes de discriminar as classes \cite{bengio2009}.
}
\meutodo{
Com o objetivo de confirmar tais hipóteses, uma das propostas é analisar as características aprendidas por uma rede neural de convolução (CNN). Essa rede permite encontrar as características mais relevantes da base de imagens, que os extratores de características canônicos não capturam. Isso porque ela possui uma hierarquia de camadas, desde a imagem original até uma etapa de classificação, com o objetivo de aprender qual o melhor processamento para as imagens de modo a melhor discriminar as classes.
}


A tarefa de classificação de imagens consiste em predizer corretamente uma imagem como pertencente a uma classe previamente determinada. Um exemplo prático é a classificação da imagem de um \textit{oceano} como parte de uma classe denominada \textit{praia}. Uma forma de definir que certa imagem pertence à uma classe é especificar todas as regras que a caracterizam.
Porém, para a maioria dos casos isso é impossível. Considere imagens coloridas, com três canais de cores e de tamanho $256\times256$ pixels onde cada um desses 65536 pixels pode ser representado por $256^3$ combinações discretas de cores. Essa complexidade pode ser reduzida ao utilizar métodos de extração de características, os quais visam representar uma imagem com um número significativamente menor de valores vetoriais. Utilizando-se tal representação, pode-se desenvolver métodos computacionais que consigam definir e identificar a qual classe pertence a imagem -- sem a necessidade de se codificar todas as regras possíveis -- por meio de algoritmos de Aprendizado de Máquina. Esses algoritmos possuem capacidade de generalização, crucial para classificar novos exemplos não contidos na base de imagens originalmente utilizada para o seu treinamento. Assim, ``aprendem'' a determinar a classe correta para as imagens de entrada. Em uma etapa posterior pode-se validar esse aprendizado, aplicando o algoritmo a novos exemplos não contidos no treinamento.

O reconhecimento de padrões em imagens possui aspectos particulares para cada aplicação. Apesar da grande variedade de extratores de características disponíveis, nem sempre é possível representar as imagens de maneira satisfatória. Isso porque existem conjuntos de características que dificultam a diferenciação entre as classes. Um dos objetivos da área de Aprendizado de Máquina é encontrar quais são essas características que melhor discriminam as classes e, dessa forma, obtêm melhores resultados na etapa de reconhecimento. É comum concentrar o maior esforço dessa tarefa ao operar no espaço de características já extraídas. Para lidar com a deficiência da extração dessas características, podem ser utilizadas transformações do espaço ou sistemas de classificação complexos. No entanto, imagens obtidas de diferentes fontes, como imagens naturais, de microscopia, telescopia e tomografia, possuem características que podem ser exploradas além dos métodos clássicos. Por isso é importante investigar métodos de processamento e preparação de imagens antes da extração dessas características, ao invés de lidar com a má representação das imagens. O uso desses métodos pode revelar \textit{características latentes}, não visíveis nas imagens originais. Tais características podem melhor descrever certas classes, pois melhoram o conjunto de representações de imagens fornecidas à etapa de classificação.

% O objetivo desta dissertação é encontrar tais características latentes que possam melhor descrever certas classes do problema.
% Um exemplo visual será apresentado na Seção \ref{sec:latentes}.

Considerando que é comum realizar a extração de características a partir da imagem original, sem preocupação com a preparação da imagem, o enfoque desse estudo é na etapa de pré-processamento, destacada na Figura~\ref{fig:fluxo}. Esta ilustra as etapas canônicas do reconhecimento de padrões desde a aquisição da imagem até sua posterior classificação. As etapas de pré-processamento e segmentação --- apresentadas em destaque --- são normalmente pouco exploradas, quando comparadas com as etapas posteriores.


\begin{figure}[!ht]
 \begin{center}
   \includegraphics[width=0.3\linewidth]{figuras/flow.png}
 \end{center}
 \caption[Etapas canônicas do reconhecimento de padrões desde a aquisição da imagem até sua posterior classificação.]{Etapas canônicas do reconhecimento de padrões desde a aquisição da imagem até sua posterior classificação. As etapas de pré-processamento e segmentação --- apresentadas em destaque --- são normalmente pouco exploradas, quando comparadas com as etapas posteriores. O enfoque desse estudo é dar maior atenção à etapa de pré-processamento. \textit{Fonte:~Elaborado pela autora.}}
 \label{fig:fluxo}
\end{figure}

\enlargethispage{-\baselineskip}

O desbalanceamento de classes também se apresenta como um obstáculo para que a classificação de imagens seja satisfatória. Esse problema é caracterizado pela diferença entre o número de exemplos disponíveis para cada classe da base de imagens. Muitos métodos de transformação do espaço de características e de classificação assumem que as classes da base estão balanceadas, o que nem sempre é verdade. Portanto, é proposto a geração de imagens artificiais a partir do processamento das imagens originais, com o objetivo de rebalancear a base de imagens e consequentemente o modelo criado para a classificação. Esse método foi comparado com o SMOTE, técnica de sobreamostragem dos vetores de características ao interpolar os exemplos mais próximos.

De maneira sumária, {\textbf esta pesquisa busca melhorar a classificação de imagens, utilizando métodos de processamento com foco na quantização de imagens e no rebalanceamento de classes antes da extração de características.} Os resultados obtidos, posteriormente apresentados na Seção \ref{cap:resultados}, demonstram o potencial do processamento de imagens antes da extração de características. Além disso, é fornecido uma visualização do espaço de características após o rebalanceamento das classes, crucial para analisar se as novas características extraídas são relevantes, ou seja, se adicionaram informações que estavam latentes ao aprendizado. Os resultados também demonstram que a quantização das imagens permite ao mesmo tempo obter vetores de características mais compactos e com maior capacidade de discriminação entre classes.

%--------------------------------------------------------------------------------
\section{Contextualização}

O grupo de pesquisa em Visualização, Imagens e Computação Gráfica (VICG), do Instituto de Ciências Matemáticas e de Computação (ICMC), tem atuado nas áreas de apoio para a classificação de coleções de imagens. Os trabalhos do grupo estão relacionados à visualização de informação com projeções multidimensionais e árvores~\cite{Joia2011}, assim como à extração de características e classificação de coleções de imagens~\cite{Paiva2011}. No que tange o processamento de imagens digitais, \citeonline{Picon2011} e \citeonline{Ponti2013} focam no pré-processamento para obter melhores resultados de classificação.

Ainda, \citeonline{Paiva2011} mostraram que os espaços de características formados por cor e textura podem ser melhorados, porém há um limite até o qual as características podem ser transformadas, ou selecionadas, de forma a garantir a discriminação entre as classes. Tal projeto atua na investigação de métodos que permitam gerar espaços de características com maior discriminação entre as classes, facilitando a classificação.

Em outros dois trabalhos relacionados é possível ver a diferença da performance para problemas de classificação de imagens. No primeiro, os autores atingem acurácia acima de 98\% na classificação de frutas após investigar alterações nos parâmetros de aquisição, realizar pré-processamento e obter a segmentação \cite{Rocha2010}. No segundo, os autores indicam que o método utilizado para obter a imagem em escala de cinza (comumente utilizada por algoritmos de extração), pode impactar significativamente a classificação final de diversas bases de imagens~\cite{Kanan2012}.

%--------------------------------------------------------------------------------
\section{Contribuições}

Conforme anteriormente mencionado, muitos aspectos influenciam a performance da classificação de coleções de imagens. É comum encontrar bases cuja extração de características é considerada difícil, ou seja, nas quais algoritmos canônicos de extração não conseguem extrair características que diferenciem bem as classes, prejudicando sua posterior classificação. Nessa situação, normalmente tenta-se lidar com as particularidades das características extraídas através de transformações no espaço de atributos ou mesmo projetando classificadores mais elaborados. Acredita-se que, ao invés disso, é importante investigar métodos de processamento e preparação de imagens antes da extração das características.
O objetivo desta pesquisa é explorar as etapas de processamento de imagens com o intuito de melhorar a discriminação entre classes de uma coleção de imagens.

Além disso, o desbalanceamento de classes é um obstáculo para uma classificação satisfatória, e por isso também será estudado. Em bases médicas, por exemplo, a quantidade de imagens relacionadas com uma doença rara é menor do que as imagens de pacientes sem a doença. Nessas situações, em que as imagens representam eventos importantes porém menos frequentes, o sistema de classificação pode ter problemas para lidar com a classe minoritária. {\textbf A hipótese, nesse caso, é que a geração de imagens artificiais como preparação para a extração de características pode melhorar a acurácia da classificação, quando comparada à geração de exemplos artificiais no espaço de atributos.} Ou seja, gerar novas imagens artificiais — que serão posteriormente reduzidas a atributos — pode apresentar melhores resultados para a classificação do que o \textit{bootstrap} de atributos artificiais.

Dados tais aspectos, pode-se então diferenciar as contribuições desta pesquisa:

\begin{description}
\item[Geral] \

Investigar os métodos de pré-processamento de imagens de forma a preparar uma coleção de imagens para a extração de características. Com isso, espera-se ao mesmo tempo obter características latentes e balancear o número de instâncias de diferentes classes.

\item[Específicas] \

  \begin{itemize}
    \item A geração de imagens artificiais utilizando métodos de processamento, como filtragem, adição de ruído e mistura pode contribuir com o balanceamento entre classes (em se tratando de problemas de classes desbalanceadas), melhorando a acurácia de algoritmos de classificação, quando comparada à geração de exemplos artificiais no espaço de atributos;
    \item A redução de cores das imagens antes da extração de características permite ao mesmo tempo obter vetores de características mais compactos e com maior capacidade de discriminação entre classes. Melhorando, assim, a classificação e diminuindo a complexidade do sistema.
  \end{itemize}
\end{description}

Considerando os objetivos aqui descritos, os resultados esperados desta pesquisa estão destacados na Seção \ref{cap:resultados}.

%--------------------------------------------------------------------------------
\section{Estrutura do documento}

O conteúdo desta dissertação está estruturado como segue.

\begin{description}
\item [Capítulo~\ref{cap:revisao}:] são conceituados os principais fundamentos necessários para o desenvolvimento desta pesquisa: pré-processamento de imagens, extração de características e desbalanceamento de classes.

\item [Capítulo~\ref{cap:quantization}:] a redução do número de intensidades de cor antes da extração de características é descrita.

\item [Capítulo~\ref{cap:metodo}:] descreve-se os métodos para a geração artificial de imagens com o objetivo de rebalancear classes.

\item [Capítulo~\ref{cap:resultados}:] os resultados alcançados são apresentados e discutidos.

\item [Capítulo~\ref{cap:conclusoes}:] reflite sobre as contribuições deste trabalho e apresenta os trabalhos futuros.

\end{description}

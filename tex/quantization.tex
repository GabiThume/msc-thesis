%%%%%%%%%%%%%%%%%%%%%%%%%%%%%%%%%%%%%%%%%%%%%%%%%%%%%%%%%%%%%%%%%%%%%%%%%%%%%%%%
\section{Considerações iniciais}

Sistemas de reconhecimento de imagens comumente utilizam uma imagem em níveis de cinza 8-bits com 256 intensidades para a extração de características. Ao aplicar a quantização (redução de cores) na etapa de pré-processamento, é esperada a redução da complexidade do vetor de características logo no ínicio, beneficiando todos os passos subsequentes.

Para analisar o impacto da utilização da quantização, são utilizados diferentes parâmetros de quantização combinados com quatro métodos de extração de cor e um de textura. Esses extratores foram escolhidos de acordo com os resultados apresentados por \citeonline{Penatti2012}.

Este capítulo trata dessa ideia de quantizar as imagens antes da extração de características, ligando com os métodos apresentados nos fundamentos.

%%%%%%%%%%%%%%%%%%%%%%%%%%%%%%%%%%%%%%%%%%%%%%%%%%%%%%%%%%%%%%%%%%%%%%%%%%%%%%%%
\section{Quantização de imagens}

O pipeline de reconhecimento de imagens envolve um passo de converter imagens coloridas em imagens com apenas um canal, obtendo uma imagem quantizada que pode ser então processada por métodos de extração de características. Dessa forma, cada imagem -- originalmente no espaço de cor RGB -- é convertida a um único canal com $C$ níveis de intensidade. Após, são utilizados os métodos apresentados na Seção \ref{sec:extracao} para extrair as características.
A Figura \ref{fig:quant:flow} ilustra esses passos, da aquisição até a classificação das imagens.

\begin{figure}[htbp]
  \begin{center}
    \centering
    \includegraphics[width=0.25\linewidth]{\detokenize{figuras/quantizacao/quantizationFlow.png}}
  \end{center}
  \label{fig:quant:flow}
\end{figure}

Cada método de quantização se comporta diferentemente para uma dada imagem RGB. Por exemplo, o método Intensidade mapeia todas as permutações dos mesmos valores em RBG para a mesma cor. Assim, produz um plano no cubo RBG conforme mostrado na Figura \ref{fig:quant:plano}. O Gleam também é similar, mas spanning uma superfície curva, dada a natureza da função gamma. O mesmo efeito é alcançado utilizando Intensidade'. Em todos os casos, o resultado é o mapeamento de características cromáticas bem diferentes em valores de intensidades similares. O método Luminância tenta melhorar isso ao adicionar um peso na combinação linear dos canais.

\begin{figure}[htbp]
  \begin{center}
    \centering
    \includegraphics[width=0.5\linewidth]{\detokenize{figuras/quantizacao/plano.png}}
  \end{center}
  \caption[Plano computado pelo método de conversão para escala de cinza Intensidade, quando um dos canais de cor possui valor 255.]{Plano computado pelo método de conversão para escala de cinza Intensidade, quando um dos canais de cor possui valor 255. \textit{Fonte:~\cite{Ponti2016}.}}
  \label{fig:quant:plano}
\end{figure}

Um exemplo das imagens obtidas após os métodos de quantização apresentados anteriormente pode ser visto na Figura \ref{fig:quant:quantizacoes}. Nesse caso é possível notar que tanto Luminância quanto MSB conseguem melhor discriminar cores. Além disso, o mapa de cores MSB obteve um maior número de cores únicas. A barra de gradientes abaixo da imagem dos pinceis demonstra como os métodos de quantização se comportam dada a variação de cor.

\begin{figure}[htbp]
  \begin{center}
    \begin{subfigure}{.3\textwidth}
      \centering
      \includegraphics[width=\linewidth]{\detokenize{figuras/quantizacao/fig_quanttest.png}}
      \caption{Original}
      \label{fig:quant:original}
    \end{subfigure}
    \begin{subfigure}{.3\textwidth}
      \centering
      \includegraphics[width=\linewidth]{\detokenize{figuras/quantizacao/fig_quantGleam.png}}
      \caption{Gleam}
    \end{subfigure}
    \begin{subfigure}{.3\textwidth}
      \centering
      \includegraphics[width=\linewidth]{\detokenize{figuras/quantizacao/fig_quantIntensity.png}}
      \caption{Intensidade'}
    \end{subfigure}
    \begin{subfigure}{.3\textwidth}
      \centering
      \includegraphics[width=\linewidth]{\detokenize{figuras/quantizacao/fig_quantLuminance.png}}
      \caption{Luminance'}
    \end{subfigure}
    \begin{subfigure}{.3\textwidth}
      \centering
      \includegraphics[width=\linewidth]{\detokenize{figuras/quantizacao/fig_quantMSB.png}}
      \caption{MSB}
    \end{subfigure}
    \hspace{0.1\textwidth}
  \end{center}
  \caption[Resultado da aplicação de métodos de quantização. A imagem original \protect\subref{fig:quant:original} resultou em versões de um canal de cor com 232 cores unicas para MSB e 184 cores para os restantes métodos.]{Resultado da aplicação de métodos de quantização. A imagem original \protect\subref{fig:quant:original} resultou em versões de um canal de cor com 232 cores unicas para MSB e 184 cores para os restantes métodos. \textit{Fonte:~\cite{Ponti2016}.}}
  \label{fig:quant:quantizacoes}
\end{figure}


% A Tabela \ref{tab:quantizacao} apresenta alguns examplos numéricos, com a saída de cada método. Nesse caso, as entradas são tuplas de valores $(R, G, B)$. Note que a correção \textit{gamma} deve ser computada em um intervalo de valores reais $0-1$, que depois é mapeado para o intervalo $0-255$.

Na Figura \ref{fig:quant:avioes} é demonstrado um exemplo de redução de cores utilizando o método MSB para um par de imagens da base de dados Caltech-101. Importante destacar que há preservação das cores, especialmente entre 64 e 256. Ao utilizar apenas 32 cores, as imagens ainda são satisfatórias mas há uma perda de informação considerável.

\begin{figure}[htbp]
  \begin{center}
    \centering
    \includegraphics[width=\linewidth]{\detokenize{figuras/quantizacao/fig_quantizationexample.jpg}}
  \end{center}
  \caption[Duas imagens da base de dados Caltech101 com variações no parâmetro de cor utilizando o método MSB. Da esquerda para a direita: imagem original 24-bits e na sequência suas versões quantizadas com: 256, 64, 32, 16 e 8 cores.]{Duas imagens da base de dados Caltech101 com variações no parâmetro de cor utilizando o método MSB. Da esquerda para a direita: imagem original 24-bits e na sequência suas versões quantizadas com: 256, 64, 32, 16 e 8 cores. \textit{Fonte:~\cite{Ponti2016}.}}
  \label{fig:quant:avioes}
\end{figure}

%%%%%%%%%%%%%%%%%%%%%%%%%%%%%%%%%%%%%%%%%%%%%%%%%%%%%%%%%%%%%%%%%%%%%%%%%%%%%%%%
\section{Considerações finais}
